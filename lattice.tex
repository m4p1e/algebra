\documentclass{article}

\usepackage{ctex}
\usepackage{tikz}
\usetikzlibrary{cd}

\usepackage{amsthm}
\usepackage{amsmath}
\usepackage{amssymb}

%\usepackage{unicode-math}


\usepackage[textwidth=18cm]{geometry} % 设置页宽=18

\usepackage{blindtext}
\usepackage{bm}
\parindent=0pt
\setlength{\parindent}{2em} 
\usepackage{indentfirst}


\usepackage{xcolor}
\usepackage{titlesec}
\titleformat{\section}[block]{\color{blue}\Large\bfseries\filcenter}{}{1em}{}
\titleformat{\subsection}[hang]{\color{red}\Large\bfseries}{}{0em}{}
%\setcounter{secnumdepth}{1} %section 序号

\newtheorem{theorem}{Theorem}[section]
\newtheorem{lemma}[theorem]{Lemma}
\newtheorem{corollary}[theorem]{Corollary}
\newtheorem{proposition}[theorem]{Proposition}
\newtheorem{example}[theorem]{Example}
\newtheorem{definition}[theorem]{Definition}
\newtheorem{remark}[theorem]{Remark}
\newtheorem{exercise}{Exercise}[section]

\newcommand*{\xfunc}[4]{{#2}\colon{#3}{#1}{#4}}
\newcommand*{\func}[3]{\xfunc{\to}{#1}{#2}{#3}}

\newcommand\Set[2]{\{\,#1\mid#2\,\}} %集合
\newcommand\SET[2]{\Set{#1}{\text{#2}}} %


\begin{document}
\title{Lattice}
\author{枫聆}
\maketitle
\tableofcontents

\newpage
\section{Ordered Sets}

\begin{definition}
\rm {\color{red} Partially ordered set} is a system $\mathcal{P} = (P,\leq)$ where $P$ is a nonempty set and $\leq$ is a binary relation on $P$ satisfying,for all $x,y,z \in P,$
\begin{enumerate}
	\item \rm {\makebox[8cm][l]{$x \leq x,$} (reflexivity) }
	\item \rm {\makebox[8cm][l]{if $x \leq y$ and $y \leq x$,then $x=y,$} (antisymmetry)}
	\item \rm {\makebox[8cm][l]{if $x \leq y$ and $y \leq z$,then $x \leq z.$} (transitivity)} 
\end{enumerate}
\end{definition}

\begin{definition}
\rm $\mathcal{C}$ is a {\color{red} chain} if for every $x,y \in \mathcal{C}$,either $x \leq y$ or $y \leq x.$
\end{definition}

{\color{blue} chain上的元素都可以相互比较}.

\begin{definition}
\rm We say that $x$ is {\color{red} covered} by $y$ in $\mathcal{P}$,written $x \prec y$,if $x \leq y$ and there is no $z \in P$ with $x \leq z \leq y$.
\end{definition}

\begin{definition}
\rm {\color{red} Hasse diagram} for a finite partially order set $\mathcal{P}$: the elements of $P$ are represented by points in the plane, and a line is drawn from $a$ up to $b$ precisely when $a \prec b$.
\end{definition}

\begin{center}
% https://tikzcd.yichuanshen.de/#N4Igdg9gJgpgziAXAbVABwnAlgFyxMJZARgBoAGAXVJADcBDAGwFcYkQAKM7gShAF9S6TLnyEUAJlLFqdJq3ZdpFPoOHY8BIuWmyGLNok47eAoSAwaxRbnvmHOZE6vOXRWlDol2Dinc7N1d3ESUgBmHwUjDn8VQIsRTRCpbxp9KMcKOLUEqw9Q1LlfaJNs2RgoAHN4IlAAMwAnCABbJB0QHAgkKRAACxh6KHZIMDYcxpa2mk6kMJp+weGCMfMJ1sR2mcQyPoGhoxGV+qb1na2ANnm9pdH4taQAdmmuxABWK8WD5buTpAAWZ5Id67T7gb7jX6IS4dF7Ahb7MG3CGTRBzGH-D4Iw4-FFoi6Ym5HED3RA9LZPEFY8GrSFnF4U+GEnHrMkvAGUpn8Sj8IA
\begin{tikzcd}
                                                 & {(1,1,1)} \arrow[ld, no head] \arrow[d, no head] \arrow[rd, no head] &                                                  \\
{(0,1,1)} \arrow[rd, no head] \arrow[d, no head] & {(1,0,1)} \arrow[ld, no head] \arrow[rd, no head]                    & {(1,1,0)} \arrow[d, no head] \arrow[ld, no head] \\
{(0,0,1)} \arrow[rd, no head]                    & {(0,1,0)} \arrow[d, no head]                                         & {(1,0,0)} \arrow[ld, no head]                    \\
                                                 & {(0,0,0)}                                                            &                                                 
\end{tikzcd}
\end{center}

\begin{definition}
\rm Given a partially order set, $f$ is a {\color{red} order preserving map} satisfying the condition $x \leq y$ implies $f(x) \leq f(y)$.
\end{definition}

\begin{definition}
\rm Given two posets $(P,\leq_S)$ and $(Q,\leq_Q)$,an {\color{red} order isomorphism} from $(P,\leq_S)$ to $(Q,\leq_Q)$ is a bijective order preserving map.
\end{definition}

\begin{definition}
\rm An {\color{red} ideal} $I$ of a partially ordered set $\mathcal{P}$ is a subset of the elements of $P$ which satisfy the property that if $x \in \mathcal{P}$ and exists $y \in I$ with $x \leq y$,then $x \in I$.
\end{definition}

{ \color{blue} 衍生自the ideal of ring,后面我们将会看见 the ideal of lattice}.

\begin{definition}
\rm Given an ordered set $\mathcal{P}=(P,\leq)$. The {\color{red} dual of $P$} is another poset $\mathcal{P}^d=(P,\leq^d)$ with the order relation defined by $x \leq^d y \iff y \leq x$. 
\end{definition}

\begin{definition}
\rm The dual notion of an ideal is called a {\color{red} filter} that $F$ is a subset of $P$ such $x \geq y \in F$ implies $x \in F$  
\end{definition}

{\color{blue} 类似的还有principle ideal和principle filter. 就是通过一个元素生成的}.

\begin{definition}
\rm The poset $\mathcal{P}$ has a {\color{red} maximum}(element) if there exists $x \in P$ such that $y \leq x$ for all $x \in P$.

An element $x \in P$ is {\color{red} maximal}	if there is no element $y \in  P$ with $x \leq y$ and $x \neq y$. 
\end{definition}

{\color{blue} maximum是一个名词表示最大值(greatest),maximal是一个形容词表示极大的意思. 在poset中可能不只有一个maximal element}.

\begin{lemma}
\rm The following are equivalent for an poset $\mathcal{P}$.

\begin{enumerate}
	\item Every nonempty subset $S \subseteq P$ contains an element minimal in $S$.
	\item $\mathcal{P}$ contains no infinite descending chain \[a_0 > a_1 > a_2 > \cdots.\]{\color{blue} 这里去掉等号是指$a_0 \neq a_1 \neq a_2 \neq \cdots$} 
	\item If \[a_0 \geq a_1 \geq a_2 \geq \cdots\] in $\mathcal{P}$,then there exists $k$ such that $a_n = a_k$ for all $n \geq k$.
\end{enumerate}
\end{lemma}

{\color{red} 这个lemma被称为descending chain condition. 对偶地也有ascending chain condition. original 'a partially ordered set $\mathcal{P}$ requires that all decreasing sequences in $\mathcal{P}$ become eventually constant'}.

\begin{proof}
(2) $\Rightarrow$ (3) 前提只存在finite descending chain. 假设(3)不成立,且$a_0 \geq a_1 \geq a_2 \geq \cdots$是infinite chain. 则对于任意的$k$,都能找到$n \geq k$使得$a_n \neq a_k$且$a_k \geq a_n$,那么$a_k > a_n$. 这样从$k=0,1,2,\cdots$开始我们每次都可以找到$a_{n_0} > a_{n_1} > \cdots$. 这样我们实际构造了一个infinite descending chain,这是和前提矛盾的. 若$a_0 \geq a_1 \geq a_2 \geq \cdots$是一个finite chain,它的最后一个元素显然是满足(3),这和假设是矛盾的. 

(3) $\Rightarrow$ (2) 也是分infinite chain和finite chain来讨论,finite是显然的,infinite的时候可以把它变成finite.

(1) $\Rightarrow$ (2) (1)前提满足下,假设(2)不成立,即$\mathcal{P}$存在infinite descending chain. 把这个chain上的元素取出来组成一个subset $S$,那么任取$a_k$都有$a_{k+1} \leq a_k$. 即找不到minimal.

(2) $\Rightarrow$ (1) (2)前提满足下,假设(1)不成立.  这里需要用一下{\color{blue} 选择公理}了,定义$S$上一个选择函数$\func{f}{S}{T}$,其中$T \subseteq S$. 让$a_0 = f(S)$,递归地定义对任意的$i \in \omega$有$a_{i+1} = f(\Set{s \in S}{s < a_i})$. 接下来让这个definition make sense,(2)前提下$S$是没有minimal, 所以$\Set{s \in S}{s \leq a_i}$不是empty set. 这样就找到了一个infinite descending chain,与假设矛盾.
(1) $\Rightarrow$ (2) $\Rightarrow$ (3)
(3) $\Rightarrow$ (2) $\Rightarrow$ (1)

{\color{blue} done well}!
\end{proof}
\end{document}