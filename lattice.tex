\documentclass{article}

\usepackage{ctex}
\usepackage{tikz}
\usetikzlibrary{cd}

\usepackage{amsthm}
\usepackage{amsmath}
\usepackage{amssymb}

%\usepackage{unicode-math}


\usepackage[textwidth=18cm]{geometry} % 设置页宽=18

\usepackage{blindtext}
\usepackage{bm}
\parindent=0pt
\setlength{\parindent}{2em} 
\usepackage{indentfirst}


\usepackage{xcolor}
\usepackage{titlesec}
\titleformat{\section}[block]{\color{blue}\Large\bfseries\filcenter}{}{1em}{}
\titleformat{\subsection}[hang]{\color{red}\Large\bfseries}{}{0em}{}
%\setcounter{secnumdepth}{1} %section 序号

\newtheorem{theorem}{Theorem}[section]
\newtheorem{lemma}[theorem]{Lemma}
\newtheorem{corollary}[theorem]{Corollary}
\newtheorem{proposition}[theorem]{Proposition}
\newtheorem{example}[theorem]{Example}
\newtheorem{definition}[theorem]{Definition}
\newtheorem{remark}[theorem]{Remark}
\newtheorem{exercise}{Exercise}[section]

\newcommand*{\xfunc}[4]{{#2}\colon{#3}{#1}{#4}}
\newcommand*{\func}[3]{\xfunc{\to}{#1}{#2}{#3}}

\newcommand\Set[2]{\{\,#1\mid#2\,\}} %集合
\newcommand\SET[2]{\Set{#1}{\text{#2}}} %


\newcommand\slattice{\mathcal{S}}
\newcommand\lattice{\mathcal{L}}


\begin{document}
\title{Lattice}
\author{枫聆}
\maketitle
\tableofcontents

\newpage
\section{Ordered Sets}

\begin{definition}
\rm {\color{red} Partially ordered set} is a system $\mathcal{P} = (P,\leq)$ where $P$ is a nonempty set and $\leq$ is a binary relation on $P$ satisfying,for all $x,y,z \in P,$
\begin{enumerate}
	\item \rm {\makebox[8cm][l]{$x \leq x,$} (reflexivity) }
	\item \rm {\makebox[8cm][l]{if $x \leq y$ and $y \leq x$,then $x=y,$} (antisymmetry)}
	\item \rm {\makebox[8cm][l]{if $x \leq y$ and $y \leq z$,then $x \leq z.$} (transitivity)} 
\end{enumerate}
\end{definition}

\begin{definition}
\rm $\mathcal{C}$ is a {\color{red} chain} if for every $x,y \in \mathcal{C}$,either $x \leq y$ or $y \leq x.$
\end{definition}

{\color{blue} chain上的元素都可以相互比较,所以它是totally ordered}.

\begin{definition}
\rm We say that $x$ is {\color{red} covered} by $y$ in $\mathcal{P}$,written $x \prec y$,if $x \leq y$ and there is no $z \in P$ with $x \leq z \leq y$.
\end{definition}

\begin{definition}
\rm {\color{red} Hasse diagram} for a finite partially order set $\mathcal{P}$: the elements of $P$ are represented by points in the plane, and a line is drawn from $a$ up to $b$ precisely when $a \prec b$.
\end{definition}

\begin{center}
% https://tikzcd.yichuanshen.de/#N4Igdg9gJgpgziAXAbVABwnAlgFyxMJZARgBoAGAXVJADcBDAGwFcYkQAKM7gShAF9S6TLnyEUAJlLFqdJq3ZdpFPoOHY8BIuWmyGLNok47eAoSAwaxRbnvmHOZE6vOXRWlDol2Dinc7N1d3ESUgBmHwUjDn8VQIsRTRCpbxp9KMcKOLUEqw9Q1LlfaJNs2RgoAHN4IlAAMwAnCABbJB0QHAgkKRAACxh6KHZIMDYcxpa2mk6kMJp+weGCMfMJ1sR2mcQyPoGhoxGV+qb1na2ANnm9pdH4taQAdmmuxABWK8WD5buTpAAWZ5Id67T7gb7jX6IS4dF7Ahb7MG3CGTRBzGH-D4Iw4-FFoi6Ym5HED3RA9LZPEFY8GrSFnF4U+GEnHrMkvAGUpn8Sj8IA
\begin{tikzcd}
                                                 & {(1,1,1)} \arrow[ld, no head] \arrow[d, no head] \arrow[rd, no head] &                                                  \\
{(0,1,1)} \arrow[rd, no head] \arrow[d, no head] & {(1,0,1)} \arrow[ld, no head] \arrow[rd, no head]                    & {(1,1,0)} \arrow[d, no head] \arrow[ld, no head] \\
{(0,0,1)} \arrow[rd, no head]                    & {(0,1,0)} \arrow[d, no head]                                         & {(1,0,0)} \arrow[ld, no head]                    \\
                                                 & {(0,0,0)}                                                            &                                                 
\end{tikzcd}
\end{center}

\begin{definition}
\rm Given a partially order set, $f$ is a {\color{red} order preserving map} satisfying the condition $x \leq y$ implies $f(x) \leq f(y)$.
\end{definition}

\begin{definition}
\rm Given two posets $(P,\leq_S)$ and $(Q,\leq_Q)$,an {\color{red} order isomorphism} from $(P,\leq_S)$ to $(Q,\leq_Q)$ is a bijective order preserving map.
\end{definition}

\begin{definition}
\rm Given two posets $(P,\leq_S)$ and $(Q,\leq_Q)$,an {\color{red} order embedding} from $(P,\leq_S)$ to $(Q,\leq_Q)$ is a both order-preserving and order-reflecting map that $x \leq y \iff f(x) \leq f(y)$.
\end{definition}

{\color{blue}相比order isomorphism而言稍微弱一点,不需要是一个surjective}.

\begin{definition}
\rm An {\color{red} ideal} $I$ of a partially ordered set $\mathcal{P}$ is a subset of the elements of $P$ which satisfy the property that if $x \in \mathcal{P}$ and exists $y \in I$ with $x \leq y$,then $x \in I$.
\end{definition}

{ \color{blue} 衍生自the ideal of ring,后面我们将会看见 the ideal of lattice}.

\begin{definition}
\rm Given an ordered set $\mathcal{P}=(P,\leq)$. The {\color{red} dual of $P$} is another poset $\mathcal{P}^d=(P,\leq^d)$ with the order relation defined by $x \leq^d y \iff y \leq x$. 
\end{definition}

\begin{definition}
\rm The dual notion of an ideal is called a {\color{red} filter} that $F$ is a subset of $P$ such $x \geq y \in F$ implies $x \in F$  
\end{definition}

{\color{blue} 类似的还有principle ideal和principle filter. 就是通过一个元素生成的}.

\begin{definition}
\rm The poset $\mathcal{P}$ has a {\color{red} maximum}(element) if there exists $x \in P$ such that $y \leq x$ for all $x \in P$.

An element $x \in P$ is {\color{red} maximal}	if there is no element $y \in  P$ with $x \leq y$ and $x \neq y$. 
\end{definition}

{\color{blue} maximum是一个名词表示最大值(greatest),maximal是一个形容词表示极大的意思. 在poset中可能不只有一个maximal element}.

\begin{lemma}
\rm The following are equivalent for an poset $\mathcal{P}$.

\begin{enumerate}
	\item Every nonempty subset $S \subseteq P$ contains an element minimal in $S$.
	\item $\mathcal{P}$ contains no infinite descending chain \[a_0 > a_1 > a_2 > \cdots.\]{\color{blue} 这里去掉等号是指$a_0 \neq a_1 \neq a_2 \neq \cdots$} 
	\item If \[a_0 \geq a_1 \geq a_2 \geq \cdots\] in $\mathcal{P}$,then there exists $k$ such that $a_n = a_k$ for all $n \geq k$.
\end{enumerate}
\end{lemma}

{\color{red} 这个lemma被称为descending chain condition(DCC). 对偶地也有ascending chain condition(ACC). original 'a partially ordered set $\mathcal{P}$ requires that all decreasing sequences in $\mathcal{P}$ become eventually constant'}.

\begin{proof}
(2) $\Rightarrow$ (3) 前提只存在finite descending chain. 假设(3)不成立,且$a_0 \geq a_1 \geq a_2 \geq \cdots$是infinite chain. 则对于任意的$k$,都能找到$n \geq k$使得$a_n \neq a_k$且$a_k \geq a_n$,那么$a_k > a_n$. 这样从$k=0,1,2,\cdots$开始我们每次都可以找到$a_{n_0} > a_{n_1} > \cdots$. 这样我们实际构造了一个infinite descending chain,这是和前提矛盾的. 若$a_0 \geq a_1 \geq a_2 \geq \cdots$是一个finite chain,它的最后一个元素显然是满足(3),这和假设是矛盾的. 

(3) $\Rightarrow$ (2) 也是分infinite chain和finite chain来讨论,finite是显然的,infinite的时候可以把它变成finite.

(1) $\Rightarrow$ (2) (1)前提满足下,假设(2)不成立,即$\mathcal{P}$存在infinite descending chain. 把这个chain上的元素取出来组成一个subset $S$,那么任取$a_k$都有$a_{k+1} \leq a_k$. 即找不到minimal.

(2) $\Rightarrow$ (1) (2)前提满足下,假设(1)不成立.  这里需要用一下{\color{blue} 选择公理}了,定义$S$上一个选择函数$\func{f}{S}{T}$,其中$T \subseteq S$. 让$a_0 = f(S)$,递归地定义对任意的$i \in \omega$有$a_{i+1} = f(\Set{s \in S}{s < a_i})$. 接下来让这个definition make sense,(2)前提下$S$是没有minimal, 所以$\Set{s \in S}{s \leq a_i}$不是empty set. 这样就找到了一个infinite descending chain,与假设矛盾.

(1) $\Rightarrow$ (2) $\Rightarrow$ (3)

(3) $\Rightarrow$ (2) $\Rightarrow$ (1)

{\color{blue} done well}!
\end{proof}

\begin{lemma}
\rm Let $\mathcal{P}$ be an poset satisfyint the DCC. If $\varphi(x)$ is statement such that
\begin{enumerate}
	\item $\varphi(x)$ holds for all minimal elements of $P$,and
	\item whenever $\varphi(y)$ holds for all $y < x$,then $\varphi(x)$ holds,
\end{enumerate}
then $\varphi(x)$ is true for every element of $P$.
\end{lemma}

{\color{blue} 这个lemma有点意思,如果对$P$上所有的minimal element $m$都有命题$\varphi(m)$成立, 且$\mathcal{P}$满足DCC. 那么再加上一个条件: 只要对任意元素$x \in P$,满足$y < x$都有$\varphi(y)$成立. 则对任意元素$x \in P$都有$\varphi(x)$成立}.

\begin{proof}
其实(1)是(2)的一个special case. 在(1)(2)hold的情况下,我们试想一下$\varphi(x)$没有被hold住的是哪些元素呢? 即对于某个$x$,存在$y < x$使得$\varphi(y)$没有被hold. 递归地,我们再去考虑这个$y$. 那么这里就存在一条descending chain在这里,由于$\mathcal{P}$是满足$DCC$,所以这个descending chain是infinite的. 这条chain的结尾显然是一个minimal element,但是它是满足$\varphi(x)$. 所以实际上是不存在这里的$x$不满足$\varphi(x)$. 
\end{proof}

\begin{definition}
\rm Let $\mathcal{P}$ be poset. Two elements $a$ and $b$ of $\mathcal{P}$ are called {\color{red} comparable} if $a \leq b$ or $a \geq b$. Otherwise, they are called {\color{red}incomparable}.
\end{definition}

{\color{blue} 可比性}.

\begin{definition}
\rm An {\color{red} antichain} in $\mathcal{P}$ is a subset $A$ of $\mathcal{P}$ in which each pair of different element are incomparable.
\end{definition}

\begin{definition}
\rm Define the {\color{red} width} of an poset $\mathcal{P}$ by
$$
	w(\mathcal{P}) = \sup\Set{|A|}{\text{A is an antichain in $\mathcal{P}$}}
$$
where $|A|$ denotes the cardinality(集合的势) of $A$. 
\end{definition}


\begin{definition}
\rm We define the {\color{red} chain-covering-number} CCN $c(\mathcal{P})$ to be the least cardinal number k, such that $P$ is a union of $k$ chains(finite) of $P$, means $P = \bigcup C_i$
\end{definition}

{\color{blue} 另一种covering number,有趣}.

\begin{lemma}
Suppose $P = \bigcup C_i$ where $i \in I$,then $ w(\mathcal{P}) \leq |I|$.
\end{lemma}

\begin{proof}
因为$ |A \cap C_i| \leq 1 $ for $i \in I$. 也就是说你把$A$里面的元素分开塞到$C_i$上,每次都只能塞一个. 那么最多你可以每个$C_i$上都塞一个. 
\end{proof}

\begin{theorem}
\rm (Dilworth ,1950) Let $\mathcal{P}$ be a finite poset. $w(\mathcal{P})$ is width. Then $\mathcal{P}$ is a union of $w(\mathcal{P})$-chains.
\end{theorem}

\begin{proof}
TODO.
\end{proof}

\newpage
\section{Semilattices, Lattices and Complete Lattices}

\subsection{Semilattice}
\begin{definition}
\rm A {\color{red} semilattice} is an algebra $\mathcal{S} = (S,*)$ satisfying, for all $x,y,z \in \slattice$,
\begin{enumerate}
	\item $x * x = x,$
	\item $x * y = y *x,$
	\item $x*(y*z) = x*(y*z).$
\end{enumerate}
where $*$ is binary operator.
{\color{red} 换句话说semilattice就是一个idempotent commutative semigroup(幂等交换半群)}. 
\end{definition}

\begin{theorem}
\rm In a semilattice $\slattice$,define $x \leq y$ if $x * y = x$. Then $(S,\leq)$ is a poset in which every pair of elements has a greater lower bound.

Conversely,given an poset $P$ with that property,define $x * y = g.l.b(x,y).$ Then $(P,*)$ is a semilattice.
\end{theorem}

{\color{blue} semilattice上弄了一个特殊的poset出来, 它最好的性质就是任意两个元素都有一个下确界}. 把$*$换成$\cap$,然后把$\leq$换成$\subseteq$,可能就很熟悉了. A semilattice with the above ordering is usually called {\color{red} meet semilattice}.
\begin{proof}
先证明这个是一个poset. 
\begin{enumerate}
	\item $x * x = x$ implies $x \ leq x,$
	\item if $x \leq y$ and $x \geq y$, then $x = x*y  = y*x = y,$
	\item if $x \leq y$ and $ y \leq z$. then $x *z = (x * y)*z = x *(y *z) = x * y = x$, so $x \leq z$.
	
这个greater lower bound就是$x * y$. 首先证明它是一个lower bound,$x *(x*y)=x*y$ and $y*(x*y) = x*y$,所以$x*y$是一个lower bound. 再来证明所有的lower bound都比它小,假设$z \leq x$和$z \leq y$,即$z$是$\{x,y\}$的一个lower bound. 那么$z * (x * y) = z * y = z$,所以$z \leq (x*y)$. 最后$x*y$的一个greater lower bound. 
\end{enumerate}
\end{proof}


对偶地,使得$x \geq y \iff x * y = x$,则称$\slattice$为是一个{\color{red} join semilattice}. 自然地在$(S,\leq)$下任意的pair都有一个least upper bound $x \vee y$.

\begin{definition}\rm A {\color{red} homomorphism} between two semilattice is a map $\func{f}{\slattice}{\mathcal{T}}$ with the property that $f(x*y) = f(x) * f(y)$. An {\color{red} isomorphism} is a homomorphism that injective and surjective.
\end{definition}

{\color{blue} nothing new}...

\begin{theorem}
\rm Let $\slattice$ be a meet semilattice. Define $\func{\phi}{S}{\mathcal{O}(S)}$ by
$$
\phi(x) = \Set{y \in S}{y \leq x}
$$
where $\mathcal{O}(S)$ is collection of all order ideals of $\slattice$. Then $\slattice$ is isomophic $(\mathcal{O}(S),\cap)$(注意这里是$\slattice$的image).
\end{theorem}

{\color{blue} 怎么感觉这些ideal都是principle ideal}.

\begin{proof}
$\cap$表示set inclusion,$\phi$是order-preserving和order-reflecting还是比较obvious. 所以$\phi$是一个order embedding of $\slattice$ into $\mathcal{O}(\slattice)$. Moreover $\phi(x \wedge y) = \phi(x) \cap \phi(y)$ because $x \wedge y$ is the greate lower bound of $\{x,y\}$,so that $z \leq x \wedge v$ if and only if $z \leq x$ and $z \leq y$. 
\end{proof}


\newpage
\subsection{Lattice}
\begin{definition}
\rm A {\color{red} lattice} is an algebra $\lattice = (L,\wedge,\vee)$ satisfying,for all $x,y,z \in L,$
\begin{enumerate}
	\item $x \wedge x = x$ and $x \vee x = x,$
	\item $x \wedge y = y \wedge x$ and $x \vee y = y \vee x,$
	\item $x \wedge (y \wedge z) = (x \wedge y) \wedge z$ and $x \vee (y \vee z) = (x \vee y) \vee z,$
	\item $x \wedge (x \vee y) = x$ and $x \vee (x \wedge y) = x.$ 
\end{enumerate}
\end{definition}

{\color{blue} 就第四个在我们眼里似乎没有那么自然,它叫absorption laws(吸收律),它在这里可以保证后面$\wedge$和$\vee$定义了相同的partial order(虽然是dual). 前三个我们知道lattice同时在两种binary operator都是semilattice,所以我们只要在lattoce上定义前面合适的partial order,它就是both meet and join semilattice}. 

\begin{theorem}
\rm In a lattice $\lattice$,define $x \leq y$ if and only if $x \wedge y = x$. Then $(L,\leq)$ is a poset in which every pair of elements has a greatest lower bound and a least upper bound.  
\end{theorem}

\begin{proof}
给定一个pair $(x,y)$. 前面已经证明了$x \wedge y$是它的一个greater lower bound. 再根据lattice definition的第四条的第一个式子,$x \vee y$是它的一个upper bound,第二式子说明当$x \geq y$时,有$x \vee y = x$,对偶地$x \vee y$是least upper bound. 

这里若$x \wedge y = x$,则$x \vee y = (x \wedge y) \vee y = y$.  类似地$x \vee y = y$,则$x \wedge y = x \wedge (x \vee y) = x$.  所以有一个很重要的结论就是{\color{blue} $x \wedge y = x \iff x \vee y = y$}.
\end{proof}

{\color{red} 类似的我们可以通过一个poset构造lattice}.

\begin{theorem}
\rm Given an poset $\mathcal{P}$ with that above property,define $x \wedge y = \sup\{x,y\}$ and $x \vee y = \inf\{x,y\}$. Then $(P,\wedge,\vee)$ is a lattice.
\end{theorem}


{\color{blue} 所以实际上lattice可以有两种定义第一种是前面的代数定义,第二种就是在poset上定义join和meet操作,这一点要清楚}.


{\color{red} the definitions of sublattice,homomorphism and isomorphism )}.

\newpage
\subsection{Complete Lattice}
\begin{definition}
\rm For a subset $A$ of a poset $P$,let $A^u$ denote the set of all upper bounds of $A$,
$$
\begin{aligned}
A^u &= \Set{x \in P}{x \geq a\ \text{for all}\ a \in A} \\
	&= \bigcap\limits_{a \in A} \uparrow a
\end{aligned}
$$
where $\uparrow a = \Set{x \in P}{x \geq a}$. Dually,$A^l$ is the set of all lower bounds of $A$,
$$
\begin{aligned}
A^l &= \Set{x \in P}{x \leq a\ \text{for all}\ a \in A} \\
	&= \bigcap\limits_{a \in A} \downarrow a
\end{aligned}	
$$
where $\uparrow a = \Set{x \in P}{x \leq a}$.
\end{definition}

思考一个问题{\color{red} poset $P$的一个subset $A$什么时候least upper bound}? 很显然$A^u$ 一定不是空的,更确切地说$A^u$有一个greatest lower upper $z$,而且$z \in A^u$,根据$z$的definition它是$A$的least upper bound. 这种情况下我们就说{\color{red} the join of $A$ exists,and write $z = \bigvee A$}. 对偶地,{\color{red} 考虑$A$的greatest lower bound},则$A^l$一定不为空,那么$A^u$里面是有一个lower upper bound的$w$,根据$w$的definition它是$A$的greatest lower bound. 这种情况下我们就说{\color{red} the meet of $A$ exists,and write $w = \bigwedge A$}.

{\color{blue} 这样我们define两个特殊的meet和join作用在一个集合上}.

\begin{theorem}
\rm Let $\slattice$ be a finite meet semilattice with greatest element 1. Then $\slattice$ is a lattice with join operation defined by 
$$
x \vee y = \bigwedge \{x,y\} ^u = \bigwedge (\uparrow x \cap \uparrow y).
$$
\end{theorem}

\begin{proof}
$\slattice$有greatest element,则$A^u$肯定不是空了,至少这个greatest element里面. $\bigwedge A^u$就是要找$A^u$的lower upper bound. 由于$\slattice$是一个finite lattice,所以$A^u$也是finite. $A^u$里面的元素做有限次meet操作得到就是一个lower upper bound,但是你还得说明它在$A^u$里面. 这是很显然的,$x \wedge z_1 \wedge \cdots \wedge z_k = x$其中$z_i \in A^u$,所以$\bigwedge A^u$是它的一个upper bound.

还得proof一下它是一个lattice,上面只是证明了这个东西是well behaved. Lattcie definition中前三条还是比较明显的.
$$
x \wedge (x \vee y) = x
$$
这也很显然,因为$x \vee y \in \{x,y\}^u$.
$$
x \vee (x \wedge y) = x
$$
因为$x \wedge y$是$\{x,y\}$的一个greatest lower bound,有$x \geq x \wedge y$,那么$\inf(x,x \wedge y) = x$.
\end{proof}

这个theorem告诉我们: {\color{red} if a finite poset $P$ has a greatest element and every pair of elements has a meet,
then $P$ is a lattice}.

\begin{theorem}
\rm Every finite subset of a lattice has a greatest lower bound and a leaset upper bound.
\end{theorem}

\begin{proof}
$\lattice$是finite,则它的subset也是finite. 前面我们知道lattice中任意一个pair都有greatest lower bound和least upper bound,这是meet和join定义下的partial order所带来的性质. 在finite subset里面先挑两个出来做meet或者join可以得到inf和sup它们也是属于$L$的,再从剩下的subset里面再挑一个出来做同样的操作,这个操作只会做有限多次,所以最终我可以得到这个subset的greatest lower bound和least upper bound.
\end{proof}

{\color{red} 这个性质在infinite lattice下可能就无法成立}. 例如infinite subset上述操作可能根本就停不下. 由此我们定义另外一个概念.

\begin{definition}
\rm Given poset $\lattice$. If every subset $A$ of $\lattice$ has a greatest lower bound $\bigwedge A$ and a least upper bound $\bigvee A$,then $\lattice$ is called {\color{red} complete lattice}. 
\end{definition}

{\color{blue} subset其中就包含了pair,所以它是一个lattice还是比较明显的}. 此外,{\color{red} finite lattice是complete的,并且所有complete lattice都包含greatest element和least element}.

\begin{definition}
\rm a {\color{red} complete meet semilattice} is an poset $\slattice$ with greatest element and the property that every nonempty subset $A$ of $S$ has a greatest lower bound $\bigwedge A$.
\end{definition}

{\color{red} 下面这个theorem让我们抛弃了更强的finite,只需要complete就可以在meet semilattice上构造一个lattice出来}.

\begin{theorem}
\rm If $\lattice$ is a complete meet semilattice,then $\lattice$ is a complete lattice with the join operation defined by
$$
\bigvee A = \bigwedge A^u = \bigwedge (\bigcap\limits_{a \in A} \uparrow a).
$$
\end{theorem}

\begin{proof}
和前面在finite meet semilattice上构造lattice类似,这里finite换成了complete. 这里我们直接就可以知道$\bigwedge A^u$是有意义的,前面也证明了它还是一个$A$的upper bound. 那么$\bigwedge A$的definition是满足$A$的least upper bound.
\end{proof}

\newpage
\subsection{Closure System}

\begin{definition}
\rm A {\color{red} closure system} on a set $X$ is a collection $\mathcal{C}$ of subsets of $X$ thats is closed under arbitrary intersections(任意的交). The sets in $\mathcal{C}$ are called closed set. 
\end{definition}

\begin{example}
有一些closure system的例子
\begin{enumerate}
	\item closed subsets of topological space,
	\item subgroups of group,
	\item subspace of vector space.
	\item convex subsets of euclidean space $\mathbb{R}^n$,
	\item order ideals of an poset.
\end{enumerate}
\end{example}

{\color{red} By convention,把$\bigcap \emptyset = X$的话,closure system就是一个complete lattice}.


\begin{definition}
\rm A {\color{red} closure} operator on a set $X$ is a map $\func{\Gamma}{\mathfrak{P}(X)}{\mathfrak{P}(X)}$ satifying, for all $A,B \subseteq X$,
\begin{enumerate}
	\item $A \subseteq \Gamma(A),$
	\item $A \subseteq B$ implies $\Gamma(A) \subseteq \Gamma(B),$
	\item $\Gamma(\Gamma(A)) = \Gamma(A).$
\end{enumerate}
\end{definition}

{\color{blue} closure的general definition有点意思}. 简单地来说,就是(1)集合的闭包包含集合本身;(2)如果两个集合有包含关系,则它们的闭包也有相同的包含关系;(3)闭包的闭包是其本身.

\begin{theorem}
\rm If $\mathcal{C}$ is a closure system on a set $X$,then the map $\func{\Gamma_{\mathcal{C}}}{\mathfrak{B}(X)}{\mathfrak{B}(X)}$ defined by 
$$
\Gamma_{\mathcal{C}}(A) = \bigcap \Set{D \in \mathcal{C}}{A \subseteq D}
$$
is a closure operator. Moreover $\Gamma_{\mathcal{C}}(A)=A$ if and only if $A \in \mathcal{C}$
\end{theorem}

{\color{red} 所有包含这个集合的closed set的交是这个集合的closure}. {\color{blue} 在topology里面closure是包含这个集合最小的closed set}. 

\begin{definition}
\rm A set of {\color{red} closure rules} on a set $X$ is a collection $\sum$ of properties $\varphi(S)$ of subsets of $X$. where each $\varphi(S)$ has one of the forms
$$
x \in S
$$
or 
$$ 
Y \subseteq S \Rightarrow z \in S
$$ 
with $x,z \in X$ and $Y \subseteq X$. A subset $D$ of $X$ is said to be closed with respect to these rules if $\varphi(D)$ is true for each $\varphi \in \sum$.
\end{definition}

{\color{blue} 你看到这里一定会感觉非常的困惑,closure rules到底是个啥东西}?

\begin{example}
\rm 对应前面列举到的closure system.
\begin{enumerate}
	\item In topological space, all rules $Y \subseteq S \Rightarrow z \in S$ where $z$ is an accumulation point of $Y$.  
	\item In subgroup, the rule $1 \in S$ and all rules
	$$
		\begin{aligned}
			x \in S &\Rightarrow x^{-1} \in S
			\{x,y\} \in S &\Rightarrow xy \in S
		\end{aligned}
	$$
	\item In vector space,$0 \in S$ and all rules $\{x,y\} \subseteq S \Rightarrow ax + by \in S$ with $a,b$ scalars. 
\end{enumerate}



\end{example}


\end{document}