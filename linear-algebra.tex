\documentclass{article}

\usepackage{ctex}
\usepackage{tikz}
\usetikzlibrary{cd}

\usepackage{amsthm}
\usepackage{amsmath}
\usepackage{amssymb}

\usepackage{unicode-math}


\usepackage[textwidth=18cm]{geometry} % 设置页宽=18

\usepackage{blindtext}
\usepackage{bm}
\parindent=0pt
\setlength{\parindent}{2em} 
\usepackage{indentfirst}


\usepackage{xcolor}
\usepackage{titlesec}
\titleformat{\section}[block]{\color{blue}\Large\bfseries\filcenter}{}{1em}{}
\titleformat{\subsection}[hang]{\color{red}\Large\bfseries}{}{0em}{}
%\setcounter{secnumdepth}{1} %section 序号

\newtheorem{theorem}{Theorem}[section]
\newtheorem{lemma}[theorem]{Lemma}
\newtheorem{corollary}[theorem]{Corollary}
\newtheorem{proposition}[theorem]{Proposition}
\newtheorem{example}[theorem]{Example}
\newtheorem{definition}[theorem]{Definition}
\newtheorem{remark}[theorem]{Remark}
\newtheorem{exercise}{Exercise}[section]

\newcommand*{\xfunc}[4]{{#2}\colon{#3}{#1}{#4}}
\newcommand*{\func}[3]{\xfunc{\to}{#1}{#2}{#3}}

\newcommand\Set[2]{\{\,#1\mid#2\,\}} %集合
\newcommand\SET[2]{\Set{#1}{\text{#2}}} %
%\newcommand\dim[1]{\text{dim}(#1)}
\newcommand\nul[1]{\text{null}\ #1}
\newcommand\range[1]{\text{range}\ #1}

\begin{document}
\title{Linear Algebra}
\author{枫聆}
\maketitle

\tableofcontents
\newpage 
\section{Vector Space}
\subsection{Definition of Vector Space}       

\begin{definition}
\rm vector spaces是一个具有加法(addition)和数量乘法(salar multiplication)的集合$V$

\begin{itemize}
	\item 加法是指对任意的元素$u,v \in V$,有$u+v \in V$
	\item 数量乘法是指对任意的元素$\lambda \in F$和$v \in V$,有$\lambda v \in V$
\end{itemize}

同时它们存在下面的属性:
\begin{itemize}
	\item 加法交换律 $\forall u,v \in V,u+v=v+u$
	\item 加法结合律 $\forall u,v,w \in V ,(u+v)+w=u+(v+w)$和$\forall a,b \in F,(ab)v=a(bv)$
	\item 加法单位元 $\forall v\in V,v+0=v$
	\item 加法逆元 $\forall v \in V,\exists v^{-1} ,v+v^{-1}=0$
	\item 数量乘法的单位元 $\forall v \in V,1v=v$
	\item 分配律 $\forall v,w \in V ,\forall a,b \in F, a(u+v)=au+av,(a+b)v=av+bv$
\end{itemize}
\end{definition}


vector spaces背后的直觉是什么?《linear algebra done right》上说来自于$F^{n}$上的addition和scalar multiplication. 很明显addition包含了abelian group的所有性质,vector space是 $R$-module的特殊化,$R$-module从结构上来说,要比Ring(带单位元)的性质要弱一些,主要体现在乘法上,弱化为数量乘法,表示把一个环作用在一个abelian group上,而不是环上的乘法. 这样做的好处是可以让可以把环上一些看起来不那么好性质都变得好一点,例如ideal在一般情况下并不一定还是一个环. 不那么好是相对于群的一些性质来说的,例如正规子群. 在定义了module之后,你会发现ideal就是一个module,我们的操作终于都在一个东西下面了,不会轻易的跑出去.

我想这里多记录一些$R$-module的东西,如何把一个环弱化成一个模结构呢?首先我们需要一个abelian group $M$,定义“the left-action of a ring $R$ on $M$”为\[\func{\sigma}{R}{\textsf{End}_{Ab}(M)}\]是一个环同态. 可能这里有一个小疑问为什么$\textsf{End}_{Ab}(M)$是一个环结构? 首先这个环里面的元素都是关于$M$的endomorphisms,乘法定义为endomorphisms之间的复合,加法定义为$(f+g)(a)=f(a)+g(a)$,其中$f,g \in \textsf{End}_{Ab}(M)$. 我们说$\sigma$把$M$变成了一个left-$R$-module,$\sigma$这个映射可以理解为left-$R$-module structure(算子),\[\sigma(r)(m) \in _{R}M\]间接的定义了数量乘法,并且有一些有趣的性质:

\begin{itemize}
	\item 分配律 $\sigma(r)(m_1+m_2)=\sigma(r)(m_1)+\sigma(r)(m_2)$ 
	\item 分配律 $\sigma(r_1+r_2)(m)=\sigma(r_1)(m)+\sigma(r_2)(m)$ (环同态下可以保证)
	\item 结合性 $\sigma(r_1r_2)(m)=\sigma(r_1)(\sigma(r_2)(m))$ (环同态下可以保证)
	\item 单位元 $\sigma(1)(m)=m$
\end{itemize}

如果要定义更清楚一点,可以这样\[\func{\rho}{R \times M}{M}\],$\rho$和$\sigma$的关系为\[\rho(r,m)=\sigma(r)(m)\],但是这里不是环同态无法保证上面的一些性质,所以需要额外规定一些东西

\begin{itemize}
	\item $\rho(r_1+r_2,m)=\rho(r_1,m)+\rho(r_2,m)$(分配律)
	\item $\rho(r,m_1+m_2)=\rho(r,m_1)+\rho(r,m_2)$(分配律)
	\item $\rho(r_1r_2,m) = \rho(r_1,\rho(r_2,m))$(结合率)
	\item $\rho(1,m) = m$ (单位元)
\end{itemize}

把vector space一般化的感觉是不是很爽?其实$R$-module在一定程度要比vector space更复杂,当用更抽象方式去理解vector space,我们定义了一个$\sigma$环同态,这个环同态很精妙的把环作用在abelian group的action表示出来,所以我们用理解R-module的方式去理解vector space,就是首先我们要有一个abelian group定义了加法,然后把一个field作用在了它之上,定义为数量乘法,最后我们就得到了这样的一个结构。

这篇note既然是在《linear algebra done right》的基础上记录的,我会尽可能的记录一些抽象的延伸的东西,让自己对linear algebra有一个不同于大学的颠覆性的认知。


\newpage
\subsection{Subspace}

\begin{definition}
如果$V$中的子集$U$是一个子空间,当且仅当满足以下条件:
\begin{enumerate}
	\item $0 \in U$
	\item $u,w \in U$蕴含$ u+w \in U$
	\item $a \in F$,$u \in U$ 蕴含$au \in U$
\end{enumerate}
\end{definition}

也就是说,$V$下子空间一定是子集,包含加法单位元,且在加法和数量乘法下封闭。

\begin{definition}
定义$U_1,\cdots,U_m$是$V$的子集.这些子集的和表示为\[U_1 + \cdots + U_m = \Set{u_1 + \cdots + u_m}{u_1 \in U_1,\cdots,u_m \in U_m}\]
\end{definition}

定义子集和,是为了引入下面这个性质

\begin{proposition}
定义$U_1,\cdots,U_m$都是$V$中的子空间,则$U_1 + \cdots + U_m$是包含$U_1,\cdots,U_m$的最小子空间.
\end{proposition}

\begin{proof}
最小的就是指$V$里面任意包含$U_1,\cdots,U_m$的子空间都包含$U_1 + \cdots + U_m$,首先得证明一下$U_1 + \cdots + U_m$是一个子空间,按照子空间的定义,加法单位元和封闭性都很显然,满足上述条件的子空间很也显然需要包含所有子空间对应的子集和。
\end{proof}

\begin{definition}
定义$U_1,\cdots,U_m$都是$V$中的子空间,如果$U_1 + \cdots + U_m$中的每个元素都有唯一分解形式即$u_1,\cdots,u_m$,则$U_1 + \cdots + U_m$是直和(direct sum),用$U_1 \oplus \cdots \oplus U_m$表示。
\end{definition}

{\color{red} 那怎么判定一个子集合是不是直和呢}?

\begin{proposition}
定义$U_1,\cdots,U_m$都是$V$中的子空间,如果$U_1 + \cdots + U_m$是直和当且仅当$u_1+\cdots+u_m = 0$时,$u_1=\ldots=u_m=0$。
\end{proposition}

即只需要$0$有唯一的表示形式就够了,来证明一下

\begin{proof}
假设\[u_1+\cdots+u_m = u_1'+\cdots+u_m'\],整理一下\[(u_1-u_1')+\cdots+(u_m-u_m')=0\],$0$有唯一的表示方式,则$u_1=u_1',\ldots,u_m=u_m'$($u_1-u_1' \in U_1,\ldots,u_m -u_m' \in U_m$)
\end{proof}

{\color{red} 下面再来个特殊情况,只有两个子空间,怎么判定它们的子集和是不是直和}?

\begin{proposition}
定义$U$和$W$是$V$中的两个子空间,如果$U+W$是直和当且仅当$U \cap W = {0}$.
\end{proposition}

\begin{proof}
\Rightarrow 如果$U+W$是直和,任取$v \in U \cap W$,则$-v \in U \cap W$,而$0 = v + (-v)$,所以$v$只能是$0$.

\Leftarrow 如果$U \cap W = \{0\}$,我们假设还有$0 = v+w$,其中$v,w$不为0,则$w=(-v)$,则$v \in U \cap W$,与前提矛盾.所以$0$有唯一表示。
\end{proof}

\newpage
\section{Finite-Dimensional Vector Space}

\subsection{Linear Combinations and Span}

\begin{definition}
\rm 给定$V$里面一系列向量$v_1,v_2,\cdots,v_m$(vector list),它们的线性组合表示为
$$
a_1v_1 + a_2v_2 + \cdots + a_mv_m,
$$
其中$a_1,a_2,\cdots,a_m \in F.$
\end{definition}

\begin{definition}
\rm 给定$V$里面一系列向量$v_1,v_2,\cdots,v_m$,所有的它们的线性组合构成的集合叫一个\textbf{span}(linear span). 
$$
\text{span}(v_1,v_2,\cdots,v_m) = \Set{a_1v_1 + a_2v_2 + \cdots + a_mv_m}{a_1,a_2,\cdots,a_m \in F}.
$$
空的vector list的span表示为$\{0\}.$
\end{definition}

{\color{red}下面是关于span一些有趣的性质}.

\begin{proposition}
\rm span是包含当前vector list里面所有vectors的smallest subspace.
\end{proposition}

\begin{proof}
首先你得证明span确实是一个subspace,然后所有其他的subspace都包含它. span是一个subspace是clearly的. 假设$U$这样一个subspace,包含$v_1,v_2,\cdots,v_m$. 则$a_1v_1,a_2v_2,\cdots,a_mv_m$也是属于$U$的. 那么它们的和也应该是属于$U$的. 这就证明了span是含于$U$的. 
\end{proof}


{\color{red} 上面span是一个名词,那么下面span就是一个动词}.

\begin{definition}
\rm 如果$\text{span}(v_1,v_2,\cdots,v_m)$等于$V$,则称$v_1,v_2,\cdots,v_m$ \textbf{spans} $V$	
\end{definition}

{ \color{red} finite-dimensional vector space的严格定义}.

\begin{definition}
\rm 如果一个vector space可以被一些vector list spans, 则称这个vector sapce是finite-dimensional.
\end{definition}

这里没有用到basis的概念...

\begin{definition}
\rm $\mathcal{P}(F)$用来表示polynomials over $F$(所有系数属于$F$的多项式集合).
\end{definition}

%https://math.stackexchange.com/questions/2612814/zero-function-implies-zero-polynomial  p(x)都等于0,则所有系数都等于0


\newpage
\subsection{Linear Independence}

\begin{definition}
\rm 给定$V$上一个vector list $v_1,v_2,\cdots,v_m.$若要使得$a_1v_1+a_2v_2+\cdots+a_mv_m=0$,只能唯一取$a_1=a_2=\cdots=a_m.$则称$v_1,v_2,\cdots,v_m$ \textbf{linearly independent}. 从另一方面说就是在$\text{span}(v_1,v_2,\cdots,v_m)$里面的所有vector都有唯一表示形式.

empty vector list同样被定义为linearly independent.
\end{definition}


{\color{red} 下面是关于linearly dependent的一个小性质}.

\begin{lemma}
\rm 给定$V$上一组linearly dependent的vector list $v_1,v_2,\cdots,v_m.$则存在一个$j \in \{1,2,\cdots,m\}$使得下面两个命题成立.
\begin{enumerate}
	\item $v_j \in \text{span}(v_1,v_2,\cdots,v_{j-1});$
	\item 如果$j^{th}$ vector从$v_1,v_2,\cdots,v_m$中被去掉,则剩下的vectors构成的span依然等于$\text{span}(v_1,v_2,\cdots,v_m).$ 
\end{enumerate}
\end{lemma}

{\color{blue} 这个lemma本质就是在linearly dependent vector list里面可以挑一个vector出来,它可以被其他的vectors表示}.

\begin{proof}
因为$v_1,v_2,\cdots,v_m$ linear independent,所以存在某个$a_k,k \in \{1,\cdots,m\}$不等于$0$,使得
$$
a_1v_1+a_2v_2+\cdot+a_mv_m=0.
$$
让$j$表示最大值形如$a_j \neq 0.$则我们有
$$
v_j = -\frac{a_1}{a_j}v_1-\frac{a_2}{a_j}v_2-\cdots--\frac{a_{j-1}}{a_j}v_{j-1}.
$$ 
\end{proof}

还有一个比较直观的判定给定vector list是不是linearly independent的方法.

\begin{proposition}
\rm linearly independent vector list长度是小于等于spanning vector list(vector list spans $V$)的长度.
\end{proposition}

\begin{proof}
给定$v_1,v_2,\cdots,v_m$ linearly independent 和 $u_1,u_2,\cdots,u_n$ spans $V$. 我们的目标是证明$m \leq n.$整个证明过程非常的有趣,会用$v_1,v_2,\cdots,v_m$替换部分$u_1,u_2,\cdots,u_n.$

首先我们考虑$v_1, u_1,u_2,\cdots u_m$,现在这个vector list变成了linearly dependent,因为$v_1$现在至少可以有两种表示方法了. 使用前面关于linearly dependent的lemma我们可以去掉某个$u_k$,使得剩下的vector list还是spans $V$. 我们称这个新的vector list为$B$.

然后by induction,我们可以把剩下的$v_2,\cdots,v_m$也都加到$B$里面,同时每次remove一个$u_k$. 我们必须考虑一下有没有足够多的$u_k$够我们remove?也就是说我们还剩下一些$v_i$放不进去$B$. 假设存在这种情况,也就是说现在$B$里面全是$v$,所以$B$里面的vectors都是linearly independent,并且spans $V$,现在还剩下$v_i$还没放进去. 矛盾就来了,$v_i$加上$B$里面的vectors就变成linearly dependent了. 所以是不存在这种情况的. 也就是说$v_1,v_2,\cdots,v_m$是完全可以一一替换部分$u_1,u_2,\cdots,u_n$的. 最后结论就是$m \leq n.$
\end{proof}


\begin{proposition}
\rm finite-dimensional vector space的subspace还是finite-dimensional.
\end{proposition}

\begin{proof}
这里还是一个构造证明,用$U$表示$V$上的subspace. 如果$U = \{0\}$,clearly. 如果$U \neq \{0\}$,则我们从里面挑一个非零vector $v_1$出来.  如果$U = \text{span}(v_1)$,we are done. 反之我们接着取非零$v_2 \in U$,这里还有有一个条件就是$v_2 \notin \text{span}(v_1).$   

一般地,如果$U = \text{span}(v_1,v_2,\cdots,v_{j-1})$,we are done. 反之我们接着取$v_j \in U$,且$v_j \notin \text{span}(v_1,v_2,\cdots,v_{j-1})$. 你可以观察到这样取出来的vector list都是linearly independent. 根据前面的lemma我们知道它的长度应该是小于任何的spanning list,$V$现在是finite-dimensional,根据定义它可以被一些vector list span. 所以最终我们这样的取法是会停下来的,因为长度被限制了,即$U$最终也会被一个vector lists span,$U$也是finite-dimensional. {\color{red} 其实感觉有点找basis的感觉了.} 
\end{proof}

\newpage
\subsection{Bases}

\begin{definition}
\rm 若$V$上的一组list of vectors是linearly independent且spans $V$,则称它是$V$上的一个basis(基).
\end{definition}


{\color{red} basis更常见的另一种描述方式如下}

\begin{proposition}
\rm 若$V$上的一组list $v_1,v_2,\cdots,v_m$ of vectors是一个basis当且仅当任意的$v \in V$都有唯一的表示(linear combination)
$$
v = a_1v_1 + a_2v2 + \cdots + a_mv_m.
$$ 
其中$a_1,a_2,\cdots,a_m \in F.$
\end{proposition}

还有一个比较显然的性质,{\color{blue} 就是任何一个spanning list 都可以被reduced成一个basis}(同样一个linearly independent vector list也可以被extended成一个basis). {\color{red} 继而finite-dimensional vector space中一定会有一个basis}.


\newpage
\subsection{Dimension}

\begin{definition}
\rm finite-dimensional vector space $V$ 的dimension为 $V$上的一个basis的长度,用$\dim{V}$表示.
\end{definition}

{\color{red} 这个定义从侧面说明basis的长度取决于$V$,而不取决于它本身}.

\begin{proposition}
\rm 给定finite-dimensional $V$上subspace $U$. 则$\dim{U} \leq \dim{V}.$
\end{proposition}

\begin{proof}
把$U$上的basis看做$V$中一个linearly independent list. 而$V$上的basis看做$V$中一个spanning list. 前面有lemma说明linearly independent list长度是小于spanning list.
\end{proof}

{\color{red} 快速根据长度来判断一个list是不是basis的方法}

\begin{proposition}
\rm 给定finite-dimensional $V$.
\begin{enumerate} 
	\item \rm linearly independent list with right length.
	\item \rm spanning list with right length.
\end{enumerate}
\rm 其中right length表示$\dim{V}$.
\end{proposition}

\begin{proposition}
\rm 给定finite-dimensional $V$上的subspace $U_1$和$U_2$. 
$$
	\dim(U_1+U_2) = \dim U_1 + \dim U_2 - \dim(U_1 \cap U_2).
$$
\end{proposition}

\begin{proof}
首先给定$U_1 \cap U_2$的basis为$u_1,u_2,\cdots,u_m$,即$\dim(U_1 \cap U_2)=m$. 在它的基础上分别extend到$U_1$和$U_2$的basis上. $U_1$的basis为$u_1,u_2,\cdots,u_m,v_1,v_2,\cdots,v_j$,$U_2$的basis为$u_1,u_2,\cdots,u_m,w_1,w_2,\cdots,w_k$. 很自然我们希望有下面等式
$$
\begin{aligned} 
\dim(U_1 \cap U_2) &= m + j + k\\
&= (m+j)+(m+k)-m\\
&= \dim U_1+\dim U_2 + \dim(U_1\cap U_2).
\end{aligned}
$$
这个等式成立的关键是要证明$u_1,u_2,\cdots,u_m,v_1,v_2,\cdots,v_j,w_1,w_2,\cdots,w_k$是$U_1 + U_2$的一个basis. 即proof
$$
a_1u_1+a_2u_2+\cdots+a_mu_m + b_1v_1+b_2v_2+\cdots+b_jv_j+c_1w_1+c_2w_2+\cdots+c_kw_k = 0.
$$
接下来就是把$w$移到一边,证明$c_1=c_2=\cdots=c_m=0$(remove v),接着证明$a$和$c$都等于$0$(linearly independent).
\end{proof}


\newpage
\section{Linear Maps}

\begin{definition}
\rm 给定两个vector space $V$和$W$,若存在一个映射$\func{T}{V}{W}$使得
\begin{enumerate}
	\item additivity(可加性): $\forall u,v \in V,T(u+v)=Tu + Tv.$
	\item homogeneity(齐次性): $\forall v \in V, T(\lambda v) = \lambda (Tv).$
\end{enumerate}
则称这个映射是一个linear maps(linear transformation)(线性映射). 也就说这个映射保留了加法运算和数量乘法. 
\end{definition}

\begin{definition}
\rm $V$和$W$之间所有linear map构成的集合记为$\mathcal{L}(V,W).$
\end{definition}

\begin{example}
\rm 有几种特殊的linear map.
\begin{enumerate}
	\item zero map: $0 \in \mathcal{L}(V,W)$定义\[0v=0\]($0$在这里表示特殊$T$下同).
	\item identity map: $I \in \mathcal{L}(V,W)$定义为\[Iv=v.\]
	\item differentiation map: $D \in \mathcal{L}(\mathcal{P}({R}),\mathcal{P}(\mathbb{R}))$定义为\[Dp = p'.\] 这里可以想一下导数的运算法则.
	\item integration map: $T \in \mathcal{L}(\mathcal{P}({R}),\mathbb{R})$定义为\[Tp = \int_{0}^{1} p(x)dx.\]
	\item from $F^n$ to $F^m$($m,n$都是正整数): 定义$A_{j,k} \in F, j = 1,\cdots,m , k = 1,\cdots,n$,然后使得$T \in \mathcal{L}(F^n,F^m)$
	$$
		T(x_1,\cdots,x_n) = (A_{1,1}x_1+\cdots+A_{1,n}x_n,A_{m,1}x_1+\cdots+A_{m,n}x_n).
	$$
\end{enumerate}
\end{example}


{\color{red} basis在经过linear map的作用会变得如何呢}?

\begin{proposition}
\rm 给定$V$上一个basis $v_1,v_2,\cdots,v_n$和$W$上一个basis $w_1,w_2,\cdots,w_n$. 则存在一个唯一的linear map $\func{T}{V}{W}$使得
$$
	Tv_j = w_j, j = 1,\cdots,n.
$$
\end{proposition}

\begin{proof}
现在给定这样的linear map $T$,我们要证明要它是唯一的,就是要说它对任意$v \in V$都有一一对应的$w \in W$. 让$v = a_1v_1+a_2v_2+\cdots+a_nv_n$,那么
$$
\begin{aligned}
T(a_1v_1+a_2v_2+\cdots+a_nv_n) &= T(a_1v_1) + T(a_2v_2) + \cdots + T(a_nv_n)\\
 &= a_1T(v_1) + a_2T(v_2) + \cdots + a_nT(v_n)\\
 &= a_1w_1 + a_2w_2 + \cdots + a_nw_n.
\end{aligned}
$$
Q.E.D.
\end{proof}

\newpage
\subsection{Algebraic Operations On $\mathcal{L}(V,W)$}

\begin{definition}
\rm 给定两个linear map $S,T \in \mathcal{L}(V,W)$和$\lambda in F$. 其加法和数量乘法定义如下. 
\begin{enumerate}
	\item sum $S+T$: $(S+T)(v) = Sv+Tv.$
	\item product $\lambda T$: $\lambda T(v) = \lambda(Tv).$
\end{enumerate}
其中$v \in V.$
\end{definition}

{\color{red} 整个$\mathcal{L}(V,W)$代数结构是一个vector space,这是一个比较重要的结论}. 当然了你可能还需要验证一下其他的axioms,例如乘法结合性,乘法单位元,分配律等等.

\begin{definition}
\rm $\mathcal{L}(V,W)$ is a vector space
\end{definition}

{\color{red} 再引入一个vector space $U$,讨论两个linear maps的product变得有意义了}.

\begin{definition}
给定$T \in \mathcal{L}(U,V)$和$S \in \mathcal{L}(V,W)$,则存在product $ST \in \mathcal{L}(U,W)$使得
$$
	(ST)(u) = S(T(u)).
$$
其中$u \in U$. 
\end{definition}

\newpage
\subsection{Null Spaces and Ranges}

\begin{definition}
\rm 给定linear map $T \in \mathcal{L}(V,W)$. 集合$\Set{v \in V}{Tv = 0}$表示\textbf{null space} of $T$,记为$\nul{T}$.
\end{definition}

{\color{red} null space 就是0的preimage收集起来}.

\begin{proposition}
\rm The null space is subspace.
\end{proposition}

\begin{proof}
我们要证明它在加法和数量乘法下封闭,并且$0$在里面. 对任意的$u,v \in \nul{T}$,有\[T(u+v) = T(u) + T(v) = 0,\]所以$u+v \in \nul{T}$. 同理$T(\lambda u) = \lambda T(u), T(0)=T(0)+T(0)=0$.
\end{proof}

{\color{red} 现在我们可以用null space来说明linear map一些性质}.

\begin{proposition}
\rm 给定linear map $T \in \mathcal{L}(V,W)$. $T$是injective当且仅当$\nul{T}=\{0\}.$
\end{proposition}

\begin{proof}
如果$T$是injective,clearly $\nul{T}=\{0\}$. 如果$\nul{T}=\{0\}$,假设$T$不是injective,则存在$a \neq b \in F$,使得$T(a) = T(b)$. 我们可以稍微变换一下$T(a-b) = T(a) - T(b)$,矛盾就来了,$(a - b) \in \nul{T}$,这是和前提矛盾的.
\end{proof}


\begin{definition}
\rm 给定映射$\func{T}{V}{W}$. 集合$\Set{Tv}{v \in V}$表示\textbf{range} of $T$,记为$\range{T}$.
\end{definition}

\begin{proposition}
\rm 若$T \in \mathcal{L}(V,W)$,则$\range{T}$是一个subspace.
\end{proposition}

\newpage
\subsection{Foudamental Theorem of Linear Maps}

\begin{theorem}
\rm 给定finite-dimensional $V$和linear map $T \in \mathcal{L}(V,W)$. 则$\range{T}$也是finite-dimensional且有下面等式成立.
$$
\dim V  = \dim \nul{T} + \dim \range{T}.
$$

\end{theorem}

\begin{proof}
在前面我们知道$\nul{T}$是一个subspace,假设它的basis为$u_1,u_2,\cdots,u_m$. 我们可以把这个basis extend到$V$的basis,即$u_1,u_2,\cdots,u_m,v_1,v_2,\cdots,v_n$. 很显然我们想要证明$\dim \range{T} = n$. 给定任意的$v \in V$,那么
$$
\begin{aligned}
 T(a_1u_1+a_2u_2+\cdots+a_mu_m + b_1v_1+b_2v_2 + \cdots + b_nv_n) &= T(a_1u_1+a_2u_2+\cdots+a_mu_m) + T(b_1v_1+b_2v_2 + \cdots + b_nv_n) \\
 &=  0 + T(b_1v_1+b_2v_2 + \cdots + b_nv_n) \\
 &= b_1Tv_1 + b_2Tv_2 + \cdots + b_nTv_n.
\end{aligned}
$$
我们可以知道$Tv_1 , Tv_2 , \cdots , Tv_n$ spans $\range{T}$,所以$\range{T}$是finite-dimensional的. 但是我们还要证明它们是$\range{T}$的一个basis, 即它们还要linearly independent. 要证明$c_1Tv_1+c_2Tv_2+\cdots+c_nTv_n=0$,那么
$$
T(c_1v_1+c_2v_2+\cdots+c_nv_n)=0.
$$
由于$v_1,v_2,\cdots,v_n$是linearly independent,所以$c_1=c_2=\cdots=c_n =0$.
\end{proof}

{\color{red} 这个theorem说明了一个重要的性质就是不存在一个injective linear map使得一个finite-dimensional vector space 到一个更低维的vector space}. {\color{blue}反之map to 一个更高维的vector space不会是一个满射}. 

\newpage
\subsection{Homogeneous system of linear equations}

\begin{proposition}
齐次线性方程组未知数的个数大于等式的个数,则有非零解.
\end{proposition}

\begin{proof}
首先得知道什么齐次线性方程组.
$$
\begin{aligned}
\sum\limits_{k=1}^{n}A_{1,k}x_k&=0 \\
&\vdots\\
\sum\limits_{k=1}^{n}A_{m,k}x_k&=0.
\end{aligned}
$$
显然$x_1=x_2=\cdots=x_n = 0$是这组方程组的解. 这是trivial的,我们要考虑的是有没有非零解的情况. 之前我们定义过一个linear map $\func{T}{F^n}{F^m}$.
$$
	T(x_1,\cdots,x_n)  = (\sum\limits_{k=1}^{n}A_{1,k}x_k,\cdots,\sum\limits_{k=1}^{n}A_{m,k}x_k).
$$
上面方程组可以用$T(x_1,\cdots,x_n)=0$来简洁的表示,注意这里的$0$是$F^m$中的加法单位元. 我们想要非零解,在这里只要保证$T$不是单射就行,即只要$\nul{T}$不是$\{0\}$. 由linear map基本定理推出性质,injective的一个充分条件是只要$n>m$就行. 这就是当前命题的最本质的刻画.
\end{proof}

{\color{red} 对于非齐次方程组也有类似的结论,但是有一些不一样}.

\begin{proposition}
对于非齐次方程组
$$
\begin{aligned}
\sum\limits_{k=1}^{n}A_{1,k}x_k&=c_1 \\
&\vdots\\
\sum\limits_{k=1}^{n}A_{m,k}x_k&=c_m.
\end{aligned}
$$
\end{proposition}
当方程组的个数大于未知数的个数时,可以找到一些$c_1,\cdots,c_m$使得方程误解.

\begin{proof}
这里就变成了$T(x_1,\cdots,x_n) =  (c_1,\cdots,c_m)$. 要使得它没有解,我们可以让$\range{T}$只是$F^m$一个subspace,即$T$不是一个surjective. 所以$n<m$就可以有一个充分条件来满足这一点.
\end{proof}

\newpage
\subsection{Matrices}

\begin{definition}
\rm 一个经典的$m \times n$ matrix.
$$
\begin{pmatrix}
A_{1,1} & \cdots & A_{1,n} \\
\vdots & & \vdots \\
A_{m,1} & \cdots & A_{m,n}
\end{pmatrix}
$$
\end{definition}

\begin{definition}
\rm 给定一个linear map $T \in \mathcal{L}(V,W)$,$V$上的一个basis $v_1,\cdots,v_n$和$W$上的一个basis $w_1,\cdots,w_m$. 用$\mathcal{M}(T)$表示$m \times n$ matrix of $T$. 其中$A_{j,k}$定义为
$$
Tv_k = A_{1,k}w_1 +  \cdots + A_{m,k}w_m.
$$
\end{definition}

这个定义的直觉是什么呢?当给定一个linear map $\func{T}{V}{W}$和$V$上一个basis $v_1,\cdots,v_n$. 那么$Tv_1,\cdots,Tv_n$是$\range{T}$的一个basis,所以它们spans $\range{T}$. 这里的想法就是把$Tv_1,\cdots,Tv_n$和$W$上的basis联系起来了.

{\color{red} matrix的加法和数量乘法定义}.

矩阵的乘法需要单独拿出来讲一下,在这里我们需要再额外定义$U$和它的一个basis $u_1,\cdots,u_p$. 给定两个linear maps $\func{T}{U}{V}$和$\func{S}{V}{W}$,它们的复合$ST$也是一个linear map. {\color{red} 那么现在有一个问题是$\mathcal{M}(ST)$是否等于$\mathcal{M}(S)\mathcal{M}(S)$}?

但是在此之前,我们并没有定义矩阵乘法. 如果我们希望上面的问题有一个positive answer,定义$\mathcal{M}(S) = A$和$\mathcal{M}(T) = C$,取$1 \leq k \leq p $,那么
$$
\begin{aligned}
(ST)u_k &= S(\sum\limits_{r=1}^{n}C_{r,k}v_r) \\
		&= \sum\limits_{r=1}^{n}C_{r,k}\sum\limits_{j=1}^{m}A_{j,r}w_j\\
		&= \sum\limits_{j=1}^{m}(\sum\limits_{r=1}^{n}A_{j,r}C_{r,k})w_j.
\end{aligned}
$$
所以$ST$也是一个$m \times p$ matrix,这个结果是非常自然的,描述basis在linear map之间传递. 所以我们由此给出一个正式的矩阵乘法的定义.

\begin{definition}
\rm 给定$ m \times n$ matrix $A$和$n \times p$ matrix $C$. 则$m \times p$ product matrix $AC$的第j行和第$k$列定义为
$$
(AC)_{j,k} =  \sum\limits_{r=1}^{n}A_{j,r}C_{r,k} .
$$
\end{definition}

{\color{blue} 可以形象的记忆就是$A$的第$j$行的元素分别乘上$C$的第$k$列的元素}.
$$
(AC)_{j,k}=A_{j,\cdot}C_{\cdot,k}
$$
其中$A_{j,\cdot}$表示第$j$行$1 \times n$矩阵,$C_{\cdot,k}$表示第$k$列$n \times 1$矩阵. 还是刻画成了两个矩阵的乘法.

\begin{proposition}
特别地,
$$
(AC)_{\cdot,k}=AC_{\cdot,k}
$$
\end{proposition}
表示$AC$的第$k$的列等于$A$乘上$C$的第$k$的列(fixed k).

\begin{proposition}
\rm 给定一个$m \times n$ matrix $A$和$n \times 1$ matrix $c = \begin{pmatrix} c_1 \\ \vdots \\ c_n \end{pmatrix}$. 那么
$$
Ac = c_1A_{\cdot,1} + \cdots + c_nA_{\cdot,n}.
$$
换句话说就是在刻画一个linear combination.
\end{proposition}

\begin{proof}
$$
\begin{aligned}
Ac &=  
\begin{pmatrix}
\left(\sum\limits_{r=1}^{n}A_{1,r}C_{r,1}\right)  \\
\vdots \\
\left(\sum\limits_{r=1}^{n}A_{n,r}C_{r,1}\right)  
\end{pmatrix} \\
&= 
\begin{pmatrix}
A_{1,1}C_{1,1} \\
\vdots \\
A_{n,1}C_{1,1} 
\end{pmatrix} + \cdots +
\begin{pmatrix}
A_{1,n}C_{n,1} \\
\vdots \\
A_{n,n}C_{n,1} 
\end{pmatrix}\\ 
&= 
C_{1,1}\begin{pmatrix}
A_{1,1} \\
\vdots \\
A_{n,1} 
\end{pmatrix} + \cdots +
C_{n,1}\begin{pmatrix}
A_{1,n} \\
\vdots \\
A_{n,n} 
\end{pmatrix}\\
\end{aligned}
$$
\end{proof}

以上都是从不同的方向去思考矩阵乘法.

\begin{example}
\rm 先给出一个classic 矩阵乘法.
$$
\begin{pmatrix}
1 & 2 \\
3 & 4 \\
5 & 6 \\
\end{pmatrix}
\begin{pmatrix}
6 & 5 & 4 & 3 \\
2 & 1 & 0 & -1
\end{pmatrix} = 
\begin{pmatrix}
10 & 7 & 4 & 1 \\
26 & 19 & 12 & 5 \\
42 & 31 & 20 & 9
\end{pmatrix}
$$
那么其中结果的第2列可以表示为
$$
\begin{pmatrix}
1 & 2 \\
3 & 4 \\
5 & 6 \\
\end{pmatrix}
\begin{pmatrix}
5 \\
1 
\end{pmatrix} = 
\begin{pmatrix}
7 \\
19 \\
31 
\end{pmatrix}
$$
还可以用linear combination的眼观来看
$$
\begin{pmatrix}
7 \\
19 \\
31 
\end{pmatrix} =
5\begin{pmatrix}
1 \\
3 \\
5 
\end{pmatrix} + 
1 \begin{pmatrix}
2 \\
4 \\
6
\end{pmatrix}
$$
\end{example}

\end{document}
