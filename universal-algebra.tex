\documentclass{article}

\usepackage{ctex}
\usepackage{tikz}
\usetikzlibrary{cd}

\usepackage{amsthm}
\usepackage{amsmath}
\usepackage{amssymb}

%\usepackage{unicode-math}


\usepackage[textwidth=18cm]{geometry} % 设置页宽=18

\usepackage{blindtext}
\usepackage{bm}
\parindent=0pt
\setlength{\parindent}{2em} 
\usepackage{indentfirst}


\usepackage{xcolor}
\usepackage{titlesec}
\titleformat{\section}[block]{\color{blue}\Large\bfseries\filcenter}{}{1em}{}
\titleformat{\subsection}[hang]{\color{red}\Large\bfseries}{}{0em}{}
%\setcounter{secnumdepth}{1} %section 序号

\newtheorem{theorem}{Theorem}[section]
\newtheorem{lemma}[theorem]{Lemma}
\newtheorem{corollary}[theorem]{Corollary}
\newtheorem{proposition}[theorem]{Proposition}
\newtheorem{example}[theorem]{Example}
\newtheorem{definition}[theorem]{Definition}
\newtheorem{remark}[theorem]{Remark}
\newtheorem{exercise}{Exercise}[section]

\newcommand*{\xfunc}[4]{{#2}\colon{#3}{#1}{#4}}
\newcommand*{\func}[3]{\xfunc{\to}{#1}{#2}{#3}}

\newcommand\Set[2]{\{\,#1\mid#2\,\}} %集合
\newcommand\SET[2]{\Set{#1}{\text{#2}}} %


\newcommand\slattice{\mathcal{S}}
\newcommand\lattice{\mathcal{L}}

\newcommand\Sg{\text{Sg}}
\begin{document}
\title{Lattice}
\author{枫聆}
\maketitle
\tableofcontents
\newpage
\section{The Elements of Universal Algebra}
\subsection{Definition and Examples of Algebras}
{\color{red}
One of the aims of universal algebra is to extract, whenever possible, the common elements
of several seemingly different types of algebraic structures}.

\begin{definition}
\rm For $A$ a nonempty set and $n$ a nonnegative integer we define $A^{0} = \{\emptyset\}$,and,for $n > 0$,$A^n$ is the set of n-tuples of elements from $A$. An {\color{red} n-ary operation} (or function) on $A$ is any function $f$ from $A^n$ to $A$; n is the {\color{red} arity} (or rank) of $f$. A {\color{red} finitary operation} is an n-ary operation,for some n. The image of $\left< a_1, \cdots , a_n \right>$ under an n-ary operation $f$ is denoted by $f(a_1,\cdots,a_n)$. An operation $f$ on $A$ is called a {\color{red} nullary operation} (or constant) if its arity is zero; it is completely determinded by the image $f(\emptyset)$ in $A$ of the only element $\emptyset$ in $A^0$, and as such it is convenient to identify it with the element $f(\emptyset)$. Thus a nullary operation is thought of as an element of $A$. An operation $f$ on $A$ is {\color{red} unary, binary, or ternay} if its arity is 1,2, or 3, respectively.
\end{definition}


\begin{definition}
\rm A {\color{red} language} (or type) of algebras is a set $\mathcal{F}$ of function symbols such a nonnegative integer $n$ is assgined to each member $f$ of $\mathcal{F}$. This integer is called the arity (or rank) of $f$, and $f$ is said to be an n-ary function symbol. The subset of n-ary function symbols in $\mathcal{F}$ is denoted by $\mathcal{F}_n$.
\end{definition}

\begin{definition}
\rm If $\mathcal{F}$ is a language of algebras then an {\color{red} algebra} $\mathbf{A}$ of type $\mathcal{F}$ is an ordered pair $\left<A,F\right>$ where $A$ is a nonempty set and $F$ is a family of finitary operations on A indexed by the language $\mathcal{F}$ such that corresponding to each n-ary function symbol $f$ in $\mathcal{F}$ there is an n-ary operation $f^\mathbf{A}$ on $A$. The set $A$ is called the {\color{red} universe} ({\color{blue} 全域}) (or underlying set) of $\mathbf{A} =\left<A,F\right>$, and the $f^\mathbf{A}$ ’s are called the {\color{red} fundamental operations} ({\color{blue}基本运算}) of $\mathbf{A}$. (In practice we prefer to write just $f$ for $f ^\mathbf{A}$ —this convention creates an ambiguity which seldom causes a problem. However, in this chapter we will be unusually careful.) If $\mathcal{F}$ is finite, say $\mathcal{F} = \{f_1,\cdots,f_k\}$, we often write $\left<A, f_1,\cdots,f_k\right>$ for $\left<A,F\right>$, usually adopting the convention:
$$
arity f_1 \geq \cdots \geq arity f_k.
$$
An algebra $\mathbf{A}$ is {\color{red} unary} if all of its operations are unary, and it is mono-unary if it has just on unary operation.
\end{definition}

{\color{blue} 抽象来说一个algebra就是一个集合和一堆operations构成的,在目前已经学到的代数中operation的arity大多数不会超过2(够modern).}

\begin{example}
\rm $\mathbf{A}$ is a {\color{red} groupoid} if it has just one binary operation; this operation is usually denoted by $+$ or $\cdot$, and we write $a+b$ or $a \cdot b$ for the image of $\left<a,b\right>$ under this operation, and call it the sum or product of $a$ and $b$, respectively.
\end{example}

\begin{definition}
\rm An alegebra $\mathbf{A}$ is {\color{red} finite} if $|\mathbf{A}|$ is finite, and trivial if $|\mathbf{A}|=1$. 
\end{definition}

\begin{example}
\rm Some well-known algebras
\begin{itemize}
	\item Group
	\item Semigroup (半群) and Monoid (幺半群)
	\item ...
\end{itemize}
\end{example}

\newpage
\subsection{Isomorphic Algebras and Subalgebras}

\begin{definition}
\rm Let $\mathbf{A}$ and $\mathbf{B}$ be two algebras of the {\color{red} same type} $\mathcal{F}$. Then a function $\func{\alpha}{A}{B}$ is an {\color{red} isomorphism} from $\mathbf{A}$ to $\mathbf{B}$ if $\alpha$ is one-to-one and onto, and for every n-ary $f \in \mathbf{F}$, and for $a_1,\cdots,a_n \in A$, we have
$$
\alpha f^\mathbf{A}(a_1,\cdots,a_n) = f^\mathbf{B}(\alpha a_1,\cdots,\alpha a_n).
$$
We say $\mathbf{A}$ is isomorphic to $B$, if there is a isomorphism from $\mathbf{A}$ to $\mathbf{B}$.
\end{definition}

{\color{blue} 老样子元素上保持bijective,对应的运算结果也保持一致}.

\begin{definition}
\rm Let $\mathbf{A}$ and $\mathbf{B}$ be two algebras of the same type. Then $\mathbf{B}$ is a {\color{red} subalgebra} of $A$ if $B \subseteq A$ and every fundamental operation of $\mathbf{B}$ is the restriction of the corresponding operation of $\mathbf{A}$, i.e., for each function symbol $f$, $f^\mathbf{B}$ is $f^\mathbf{A}$ restricted to $B$; we write simply $\mathbf{B} \leq \mathbf{A}$.

A {\color{red} subuniverse} of $\mathbf{A}$ is a subset $B$ of $A$ which is closed under the fundamental operations of $\mathbf{A}$, i.e., if $f$ is a fundamental n-ary operation of $\mathbf{A}$ and $a_1,\cdots,a_n \in B$ we wound required $f(a_1,\cdots,a_n) \in B$. 
\end{definition}

{\color{blue} 这个restriction就是只对定义域做限制的意思,subalgebra是一种构造新代数的方法,后面会学到另外几种}.

\begin{definition}
\rm Let $\mathbf{A}$ and $\mathbf{B}$ be of the same type. A function $\func{\alpha}{A}{B}$ is an embedding of $\mathbf{A}$ into $\mathbf{B}$ if $\alpha$ is one-to-one and satisfies
$$
\alpha f^\mathbf{A}(a_1,\cdots,a_n) = f^\mathbf{B}(\alpha a_1,\cdots,\alpha a_n).
$$
Such an $\alpha$ is also called a monomorphism. We say $\mathbf{A}$ can be embedded in $\mathbf{B}$ if there is an embedding of $\mathbf{A}$ into $\mathbf{B}$.
\end{definition}

{\color{blue} 单纯地去掉surjective.}

\begin{theorem}
If $\func{\alpha}{A}{B}$ is an embedding, then $\alpha(A)$ is a subuniverse of $\mathbb{B}$.
\end{theorem}

\begin{proof}
我们需要说$\alpha(A)$在n-ary operation下保持封闭. 因为$\alpha$是应embedding,所以对应一个n-ary operation $f$和$a_1,\cdots,a_n \in A$有
$$
f^\mathbf{B}(\alpha a_1,\cdots,\alpha a_n) = \alpha f^\mathbf{A}(a_1,\cdots,c_n) \in \alpha(A).
$$
已经证闭.
\end{proof}

\newpage
\subsection{Algebraic Lattices and Subuniverses}

{\color{red} 这一章阐述algebraic lattice出现在universe algebra的原因}.

\begin{definition}
\rm Given an algebra $\mathbf{A}$ define, for every $X \subseteq A$,
$$
\Sg(X) = \bigcap \Set{B}{X \subseteq B\ \text{and}\ B\ \text{is a subuniverse of}\ \mathbf{A}}.
$$ 
We read $\Sg(X)$ as "{\color{red} the subuniverse generated by $X$}".
\end{definition}

\begin{definition}
\rm A closure operator $C$ on the set $A$ is an {\color{red} algebraic closure  operator} if for $X \subseteq A$
$$
C(X) = \bigcup\Set{C(Y)}{Y \subseteq X\ \text{and}\ Y\ \text{is finite}}.
$$
\end{definition}

\begin{theorem}
\rm If we are given an algebra $\mathbf{A}$, then $\Sg$ is an algebraic closure operator on $A$.
\end{theorem}

\begin{proof}
很明显任意地subuniverses交还是一个subuniverse,所有的subuiverses构成了一个closure system,所以$\Sg$是一个closure operator. 对于任意的$X \subseteq A$我们定义
$$
E(X) = X \cup \Set{f(a_1,\cdots,a_n)}{f\ \text{is a fundamental $n$-ary operation on}\ A\ \text{and}\ a_1,\cdots,a_n \in X}. 
$$
然后定义它的$n$次复合$E^n(X)$为
$$
\begin{aligned}
E^0(X) &= X \\
E^{n+1}(X) &= E(E^n(X)).
\end{aligned}
$$
由于$A$上所有fundamental operation都是finitary,且有
$$
X \subseteq E(X) \subseteq \cdots \subseteq E^n(X).
$$
接下来我们来证明下面的式子
$$
\Sg(X) = X \cup E(X) \cup E^2(X) \cup \cdots.
$$
思路是(1)$\Sg(X) \subseteq X \cup E(X) \cup E^2(X) \cup \cdots$和(2)$X \cup E(X) \cup E^2(X) \cup \cdots \subseteq \Sg(X)$. 

(1) 我们从$X \cup E(X) \cup E^2(X) \cup \cdots$这种构造出发,实际上这种构造得到的就是包含$X$的最小subuniverse. 考虑它的bound,可能存在某个最小的$k$使得$E^{k+1}(E^k(X)) = E^k(X)$,那么当$i > k$时都有$E^i(X) = E^k(X)$,无疑问现在$E^k(X)$是一个subuniverse并且包含$X$,这其实就说明了$\Sg(X) \subseteq E^k(X)$. 如果你不能找到这样的$k$,那么会有若$i>k$,那么$E^k(X) \subset E^i(X)$,但是它是bound by $A$,且$\Sg(A) = A$,这是显然地.

(2) 任取$x \in \Sg(X)$, 我们用$\{Z_i\}_{i \in I}$表示所有such that $X \subseteq Z$且$Z$是$\mathbf{A}$的一个subuniverse. 那么对任意的$i$,有$x \in Z_i$. 那么我们更有
$$
\Sg(X) = E^n(Z_1) \cap E^n(Z_2) \cap \cdots.
$$
对任意的$n$成立. 并且我们还有
$$
E^n(X) \subseteq E^n(Z_1) \cap E^n(Z_2) \cap \cdots.
$$
这是因为每一项都有$E^n(X) \subseteq E^n(Z_i)$. 那么我们把$n$取遍,就可以得到
$$
X \cup E(X) \cup E^2(X) \cup \cdots \subseteq \Sg(X).
$$
综上$\Sg(X) = X \cup E(X) \cup E^2(X) \cup \cdots.$

那么对于任意的$a \in \Sg(X)$,都有$a \in E^n(X)$其中$n < \omega$ (omega 表示无穷远它比任何的自然数都大). 那么存在finite $Y \subseteq X$,使得$a \in E^n(X)$. 这个过程就是把生成$a$的哪些元素在$E^{n-1}(X)$里面挑出来,然后循环往复,$n$是finite的,所以$Y$也是有限的. 这就说明了$\Sg(X) = \bigcap \Sg(Y_i)$,因为每一个$a$都可以找到一个finite $Y$.
\end{proof}

\begin{corollary}
\rm If $\mathbf{A}$ is an algebra then $\mathcal{L}_{Sg}$, the lattice of subuniverses of $\mathbf{A}$, is an algebraic lattice.
\end{corollary}

\end{document}