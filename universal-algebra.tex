\documentclass{article}

\usepackage{ctex}
\usepackage{tikz}
\usetikzlibrary{cd}

\usepackage{amsthm}
\usepackage{amsmath}
\usepackage{amssymb}

%\usepackage{unicode-math}


\usepackage[textwidth=18cm]{geometry} % 设置页宽=18

\usepackage{blindtext}
\usepackage{bm}
\parindent=0pt
\setlength{\parindent}{2em} 
\usepackage{indentfirst}


\usepackage{xcolor}
\usepackage{titlesec}
\titleformat{\section}[block]{\color{blue}\Large\bfseries\filcenter}{}{1em}{}
\titleformat{\subsection}[hang]{\color{red}\Large\bfseries}{}{0em}{}
%\setcounter{secnumdepth}{1} %section 序号

\newtheorem{theorem}{Theorem}[section]
\newtheorem{lemma}[theorem]{Lemma}
\newtheorem{corollary}[theorem]{Corollary}
\newtheorem{proposition}[theorem]{Proposition}
\newtheorem{example}[theorem]{Example}
\newtheorem{definition}[theorem]{Definition}
\newtheorem{remark}[theorem]{Remark}
\newtheorem{exercise}{Exercise}[section]

\newcommand*{\xfunc}[4]{{#2}\colon{#3}{#1}{#4}}
\newcommand*{\func}[3]{\xfunc{\to}{#1}{#2}{#3}}

\newcommand\Set[2]{\{\,#1\mid#2\,\}} %集合
\newcommand\SET[2]{\Set{#1}{\text{#2}}} %


\newcommand\slattice{\mathcal{S}}
\newcommand\lattice{\mathcal{L}}
\begin{document}
\title{Lattice}
\author{枫聆}
\maketitle
\tableofcontents
\newpage
\section{The Elements of Universal Algebra}

{\color{red}
One of the aims of universal algebra is to extract, whenever possible, the common elements
of several seemingly different types of algebraic structures}.

\begin{definition}
\rm For $A$ a nonempty set and $n$ a nonnegative integer we define $A^{0} = \{\emptyset\}$,and,for $n > 0$,$A^n$ is the set of n-tuples of elements from $A$. An {\color{red} n-ary operation} (or function) on $A$ is any function $f$ from $A^n$ to $A$; n is the {\color{red} arity} (or rank) of $f$. A {\color{red} finitary operation} is an n-ary operation,for some n. The image of $\left< a_1, \cdots , a_n \right>$ under an n-ary operation $f$ is denoted by $f(a_1,\cdots,a_n)$. An operation $f$ on $A$ is called a {\color{red} nullary operation} (or constant) if its arity is zero; it is completely determinded by the image $f(\emptyset)$ in $A$ of the only element $\emptyset$ in $A^0$, and as such it is convenient to identify it with the element $f(\emptyset)$. Thus a nullary operation is thought of as an element of $A$. An operation $f$ on $A$ is {\color{red} unary, binary, or ternay} if its arity is 1,2, or 3, respectively.
\end{definition}


\begin{definition}
\rm A {\color{red} language} (or type) of algebras is a set $\mathcal{F}$ of function symbols such a nonnegative integer $n$ is assgined to each member $f$ of $\mathcal{F}$. This integer is called the arity (or rank) of $f$, and $f$ is said to be an n-ary function symbol. The subset of n-ary function symbols in $\mathcal{F}$ is denoted by $\mathcal{F}_n$.
\end{definition}

\begin{definition}
\rm If $\mathcal{F}$ is a language of algebras then an {\color{red} algebra} $\mathbf{A}$ of type $\mathcal{F}$ is an ordered pair $\left<A,F\right>$ where $A$ is a nonempty set and $F$ is a family of finitary operations on A indexed by the language $\mathcal{F}$ such that corresponding to each n-ary function symbol $f$ in $\mathcal{F}$ there is an n-ary operation $f^\mathbf{A}$ on $A$. The set $A$ is called the {\color{red} universe} ({\color{blue} 全域}) (or underlying set) of $\mathbf{A} =\left<A,F\right>$, and the $f^\mathbf{A}$ ’s are called the {\color{red} fundamental operations} ({\color{blue}基本运算}) of $\mathbf{A}$. (In practice we prefer to write just $f$ for $f ^\mathbf{A}$ —this convention creates an ambiguity which seldom causes a problem. However, in this chapter we will be unusually careful.) If $\mathcal{F}$ is finite, say $\mathcal{F} = \{f_1,\cdots,f_k\}$, we often write $\left<A, f_1,\cdots,f_k\right>$ for $\left<A,F\right>$, usually adopting the convention:
$$
arity f_1 \geq \cdots \geq arity f_k.
$$
An algebra $\mathbf{A}$ is {\color{red} unary} if all of its operations are unary, and it is mono-unary if it has just on unary operation.
\end{definition}

{\color{blue} 抽象来说一个algebra就是一个集合和一堆operations构成的,在目前已经学到的代数中operation的arity大多数不会超过2(够modern).}

\begin{example}
\rm $\mathbf{A}$ is a {\color{red} groupoid} if it has just one binary operation; this operation is usually denoted by $+$ or $\cdot$, and we write $a+b$ or $a \cdot b$ for the image of $\left<a,b\right>$ under this operation, and call it the sum or product of $a$ and $b$, respectively.
\end{example}

\begin{definition}
\rm An alegebra $\mathbf{A}$ is {\color{red} finite} if $|\mathbf{A}|$ is finite, and trivial if $|\mathbf{A}|=1$. 
\end{definition}

\begin{example}
\rm Some well-known algebras
\begin{itemize}
	\item Group
	\item Semigroup (半群) and Monoid (幺半群)
\end{itemize}
\end{example}
\end{document}