\documentclass{article}

\usepackage{ctex}
\usepackage{tikz}
\usetikzlibrary{cd}

\usepackage{amsthm}
\usepackage{amsmath}
\usepackage{amssymb}

\usepackage{unicode-math}


\usepackage[textwidth=18cm]{geometry} % 设置页宽=18

\usepackage{blindtext}
\usepackage{bm}
\parindent=0pt
\setlength{\parindent}{2em} 
\usepackage{indentfirst}


\usepackage{xcolor}
\usepackage{titlesec}
\titleformat{\section}[block]{\color{blue}\Large\bfseries\filcenter}{}{1em}{}
\titleformat{\subsection}[hang]{\color{red}\Large\bfseries}{}{0em}{}
%\setcounter{secnumdepth}{1} %section 序号

\newtheorem{theorem}{Theorem}[section]
\newtheorem{lemma}[theorem]{Lemma}
\newtheorem{corollary}[theorem]{Corollary}
\newtheorem{proposition}[theorem]{Proposition}
\newtheorem{example}[theorem]{Example}
\newtheorem{definition}[theorem]{Definition}
\newtheorem{remark}[theorem]{Remark}
\newtheorem{exercise}{Exercise}[section]

\newcommand*{\xfunc}[4]{{#2}\colon{#3}{#1}{#4}}
\newcommand*{\func}[3]{\xfunc{\to}{#1}{#2}{#3}}

\newcommand\Set[2]{\{\,#1\mid#2\,\}} %集合
\newcommand\SET[2]{\Set{#1}{\text{#2}}} %

\begin{document}
\title{Linear Algebra}
\author{枫聆}
\maketitle

\tableofcontents

\newpage	
\section{Group}


\newpage
\section{Ring}

\begin{definition}
\rm 在集合$R$上定义两种二元运算操作$+$和$\cdot$,并且满足以下条件
\begin{enumerate}
	\item $R$在$+$下是一个abelian group:
	\begin{itemize}
		\item 加法结合律: $\forall a,b,c \in R, (a+b)+c = a+(b+c).$
		\item 加法交换律: $\forall a,b \in R, a+b=b+a.$
		\item 加法零元: $\forall a \in R, a+0 = a.$
		\item 加法逆元: $\forall a \in R, \exists -a \in R,  a+(-a)=0.$
	\end{itemize}
	\item $R$在$\cdot$下是一个monoid:
	\begin{itemize}
		\item 乘法结合律: $\forall a,b,c \in R, (a\cdot b)\cdot c = a \cdot (b \cdot c).$
		\item 乘法单位元: $\forall a \in R, 1 \cdot a = a , a \cdot 1 = a.$
	\end{itemize}
	\item 乘法分配率
	\begin{itemize}
		\item 左分配: $\forall a,b,c \in R , a\cdot (b+c) = (a\cdot b) + (a \cdot c).$
		\item 右分配: $\forall a,b,c \in R , (b+c) \cdot a = (b \cdot a)+(c \cdot a).$
	\end{itemize}
\end{enumerate}
则称$R$是一个\textbf{ring},上面条件称为\textbf{ring axioms}.
\end{definition}

\newpage
\section{Module}
%https://sites.math.washington.edu/~smith/Teaching/504/rings.pdf
\begin{definition}
\rm 给定一个带乘法单位元的ring $R$. 一个$\textbf{left}$ $R$ $\textbf{-module}$ $M$由一个abelian group $(M,+)$ 和 一个操作$\cdot \colon R \times M \rightarrow M$组成,并且对于任意的$r,s \in R$和任意的$x,y \in M$满足以下条件:

\begin{enumerate}
	\item $r \cdot (x+y) = r \cdot x + r \cdot y.$ 
	\item $(r+s) \cdot x = r \cdot x + s \cdot x.$
	\item $(rs) \cdot x = r \cdot (s \cdot x).$
	\item $1 \cdot x  = x.$
\end{enumerate}
类似也有\textbf{right $R$-module}.
\end{definition}

{ \color{red} 关于module更形象的理解可以去看看写在linear algebra最前面的东西}.

%https://en.wikipedia.org/wiki/Module_(mathematics)#:~:text=Such%20a%20representation%20R%20%E2%86%92,M%2C%20then%20r%20%3D%200.
{ \color{blue} If $M$ is a left $R$-module, then the action of an element $r$ in $R$ is defined to be the map $M \rightarrow M$ that sends each $x$ to $rx$ (or $xr$ in the case of a right module), and is necessarily a group endomorphism of the abelian group $(M, +)$. The set of all group endomorphisms of $M$ is denoted $End_Z(M)$ and forms a ring under addition and composition, and sending a ring element $r$ of $R$ to its action actually defines a ring homomorphism from $R$ to $End_Z(M)$}.

注意这里的环同态是在确定$M$是一个module的情况下反过来推.


\end{document}